\documentclass[a4j,twocolumn]{jsarticle}
\usepackage{amsmath}
\usepackage{svg}
\usepackage{float}
\usepackage{svgcolor}
\usepackage{graphicx} % Required for inserting images
\usepackage{booktabs}
\usepackage{pdfpages}
\usepackage{bm}
\usepackage[subrefformat=parens]{subcaption}
% コード
\usepackage{listings,jvlisting}
% 背景色やテキスト色など、VSCodeに近い見た目に設定
\definecolor{commentGreen}{rgb}{0.12,0.49,0.14}
\definecolor{stringPurple}{rgb}{0.65, 0.12, 0.82}
\definecolor{keywordBlue}{rgb}{0.11,0.35,0.69}
\definecolor{basicBlack}{rgb}{0.0, 0.0, 0.0}
\definecolor{lineNumbers}{rgb}{0.5,0.5,0.5}

\lstdefinestyle{mystyle}{
    backgroundcolor=\color{white},
    basicstyle=\footnotesize\ttfamily\color{basicBlack},
    commentstyle=\color{commentGreen},
    keywordstyle=\color{keywordBlue},
    numberstyle=\tiny\color{lineNumbers},
    stringstyle=\color{stringPurple},
    breakatwhitespace=false,         
    breaklines=true,                 
    captionpos=b,                    
    keepspaces=true,                 
    numbers=left,                    
    numbersep=5pt,                  
    showspaces=false,                
    showstringspaces=false,
    showtabs=false,                  
    tabsize=2
}

\lstset{style=mystyle}
\renewcommand{\lstlistingname}{Code}

\title{scikit-learn、Keras、TensorFlowによる実践機械学習}
\author{東京大学 工学部 3年 阿部 慧人}
\date{2024年 8月 18日}

\begin{document}

\begin{abstract}
インターンの課題図書の4章から9章の演習問題を自分で解いたものの答えを載せる。
\end{abstract}

\maketitle
\section{4章 モデルの訓練}
\subsection{問題4.1}
数百万個もの特徴量を持つ訓練セットに対して、有効な線形回帰訓練アルゴリズムは、
バッチGD、確率的GD、ミニバッチGDである。これらの勾配降下法は特徴量の数$n$に対して、$O(n)$でしか計算量が
増加しない。一方で、正規方程式やSVDは、$O(n^{2})$以上の計算量がかかる。
\subsection{問題4.2}
特徴量のスケールが大きく異なる場合、勾配降下法を使うと時間がかかるというデメリットがある。
これは、損失関数の等高線が平たい形になり直線的に最小値に移動できないことが原因である。

\subsection{問題4.3}
ロジスティック回帰モデルの損失関数は、式(1)(2)である。
\begin{gather}
    J(\theta) = -\frac{1}{m}\Sigma[y^{(i)}log(p^{(i)}) + (1-y^{(i)})log(1-p^{(i)})] \\
    where  p = \sigma(\bm{x}^{T}\theta)
\end{gather}
この式から、$J(\theta)$は、$\theta$について凸関数とわかるので、極小値は唯一でそれを最小値とすればよい。

\subsection{問題4.4}
十分な時間を�